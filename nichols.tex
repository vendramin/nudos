\documentclass[graybox]{svmult}

\usepackage{amssymb}
\usepackage{amstext}
\usepackage{mathtools}
\usepackage{mathptmx}       
\usepackage{helvet}        
\usepackage{courier}      
\usepackage{type1cm}     
\usepackage{listings}                        
\usepackage{makeidx}    
\usepackage{graphicx}   
\usepackage{multicol}        
\usepackage[bottom]{footmisc}

\newcommand{\N}{\mathbb{N}}
\newcommand{\B}{\mathbb{B}}
\newcommand{\Q}{\mathbb{Q}}
\newcommand{\Z}{\mathbb{Z}}
\newcommand{\F}{\mathbb{F}}
\newcommand{\R}{\mathbb{R}}
\newcommand{\C}{\mathbb{C}}
\newcommand{\K}{\mathbb{K}}
\newcommand{\T}{\mathbb{T}}

\newcommand{\FK}{\mathcal{E}}
\newcommand{\NA}{\mathfrak{B}}
\newcommand{\ad}{\mathrm{ad}}
\newcommand{\charK}{\mathrm{char}{\K}}
\newcommand{\supp}{\mathrm{supp}}
\newcommand{\ydG}{\prescript{G}{G}{\mathcal{YD}}}
\newcommand{\ydH}{\prescript{H}{H}{\mathcal{YD}}}

%%% Definitions
\newcommand{\Aff}{\mathrm{Aff}}
\newcommand{\Rad}{\mathrm{rad}}
\newcommand{\Inn}{\mathrm{Inn}}
\newcommand{\Stab}{\mathrm{Stab}}
\newcommand{\Ext}{\mathrm{Ext}}
\newcommand{\Img}{\mathrm{Img}}
\newcommand{\Conj}{\mathrm{Conj}}
\newcommand{\fix}{\operatorname{fix}}
\newcommand{\soc}{\operatorname{soc}}
\newcommand{\id}{\operatorname{id}}
\newcommand{\op}{\operatorname{op}}
\newcommand{\cop}{\operatorname{cop}}
\newcommand{\rk}{\operatorname{rk}}
\newcommand{\Aut}{\operatorname{Aut}}
\newcommand{\End}{\operatorname{End}}
\newcommand{\Irr}{\operatorname{Irr}}
\newcommand{\img}{\operatorname{img}}
\newcommand{\GL}{\mathbf{GL}}
\newcommand{\SL}{\mathbf{SL}}
\newcommand{\PGL}{\mathrm{PGL}}
\newcommand{\PSL}{\mathrm{PSL}}
\newcommand{\PSU}{\mathrm{PSU}}
\newcommand{\Sz}{\mathrm{Sz}}
\newcommand{\Ree}{\mathrm{Ree}}
\newcommand{\Sp}{\mathrm{Sp}}
\newcommand{\Alt}{\mathbb{A}}
\newcommand{\Sym}{\mathbb{S}}
\newcommand{\ord}{\mathrm{ord}}


\makeindex             
                      
\begin{document}

\lstset{language=GAP,
  showstringspaces=false,
  xleftmargin=0.0cm,
  xrightmargin=0.0cm,
  basicstyle=\small\ttfamily,
  frame=single,
  framerule=0pt,
}


\title*{Nichols algebras over non-abelian groups}
\author{Leandro Vendramin}

\institute{Name of First Author \at Name, Address of Institute, \email{lvendramin@dm.uba.ar}}

% Use the package "url.sty" to avoid
% problems with special characters
% used in your e-mail or web address

\maketitle

\abstract{
	These notes correspond to a mini-course given in the University of
	Edinburgh, Edinburgh, Scotland, in April 2017, and  in the workshop ``Tensor categories, Hopf
	algebras and quantum groups'', in Marburg, Germany, in January 2018.
}

\section{Braided vector spaces}
\label{sec:BVS}

We start with the definition of Nichols algebras. For that purpose we need
first to define braided vector spaces. We will work over the field $\C$ of complex numbers.

\begin{definition}
	A \emph{braided vector space} is a pair $(V,c)$, where $V$ is a vector
	space and $c\in\GL(V\otimes V)$ is a
	solution of the \emph{braid equation}:
	\[
	(c\otimes\id)(\id\otimes c)(c\otimes\id)
	=(\id\otimes c)(c\otimes\id)(\id\otimes c).
	\]
\end{definition}

\begin{example}[braided vector spaces of diagonal type]
	Let $V$ be a vector space with basis $x_{1},x_{2},...,x_{n}$. For $i,j\in\{1,\dots,n\}$ let
	$q_{ij}\in\C^\times$ and 
	let 
	\[
		c(x_{i}\otimes x_{j})=q_{ij}x_{j}\otimes x_{i}.  
	\]
	Then
	$(V,c)$ is a braided vector space.
\end{example}

\begin{example}
	Let $G$ be a finite group and $V=\C G$ be the complex vector space with basis
	$\{g:g\in G\}$. Let \[
		c(g\otimes h)=ghg^{-1}\otimes g
	\]
	for $g,h\in G$.
	Then $(V,c)$ is a braided vector space.
\end{example}

\begin{example}
	Let $V$ be a vector space with basis $x,y$. For $q\in\C^{\times}$ let $c$ be such that 
	\begin{align*}
		&c(x\otimes x)=x\otimes x, && 
		c(y\otimes y)=y\otimes y, \\
		&c(x\otimes y)=q y\otimes x, && 
		c(y\otimes x)=q x\otimes y+(1-q^2)y\otimes x.
	\end{align*}
	Then $(V,c)$ is a braided vector space.
\end{example}

We will work with a particular family of braided vector spaces.

\begin{definition}
	Let $G$ be a group. 
	A \textbf{Yetter-Drinfeld module} over $G$ is a $\C G$-module
	$V=\oplus_{g\in G}V_{g}$ such that $hV_{g}\subseteq V_{hgh^{-1}}$ for all
	$g,h\in G$. 
\end{definition}

\begin{exercise}
	Let $G$ be a group and $V$ be a Yetter-Drinfeld module over $G$. Prove that
	the map $c:V\otimes V\to V\otimes V$ given by  $c(u\otimes v)=xv\otimes u$
	whenever $\deg u=x$, is a solution of the braid equation.
\end{exercise}

Over the complex numbers, the category of Yetter-Drinfeld modules over $G$
is semisimple.  Furthermore, simple Yetter-Drinfeld modules over $G$ are
parametrized by pairs $(g^G,\rho)$, where $g^G$ denotes the conjugacy class of $g\in G$  
and $(\rho,M)$ is an irreducible representation of the centralizer
$C_G(g)$. 

Let us describe the simple Yetter-Drinfeld modules over $G$. Let $\{x_1,\dots,x_n\}$
be a set of representatives of left cosets of $C_G(g)$ in $G$.  Then the simple
Yetter-Drinfeld modules over $G$ are
\[
M(g^G,\rho)=\mathrm{Ind}_{C_G(g)}^G\rho=\bigoplus_{i=1}^n x_i\otimes_{\mathbb{C}C_G(g)}M
\]
with the induced action $y(x\otimes m)=yx\otimes m$ for $x,y\in G$ and $m\in
M$, and the coaction $\delta(x_i\otimes m)=x_igx_i^{-1}\otimes(x_i\otimes
m)$ for $m\in W$. The braiding is
\[
c\left((x_i\otimes m)\otimes(x_j\otimes m') \right)=(x_igx_i^{-1}x_j\otimes m')\otimes(x_i\otimes m).
\]
Thus every simple Yetter-Drinfeld module over $G$ can be written as
$V=\oplus_{x\in g^G}V_x$, where $V_x=\{v\in V:\delta(v)=x\otimes v\}$ and
$V_g=1\otimes M$. For all $x\in g^G$, $V_x$ is a simple $C_G(x)$-module and
$yV_x\subseteq V_{yxy^{-1}}$ for all $y\in G$.

%\section{Racks}
%
%Let us study the combinatorics behind a Yetter-Drinfeld module. 
%
%\begin{definition}
%	A \emph{rack} is a pair $(X,\triangleright)$, where $X$ is a set and
%	$X\times X\to X$, $(x,y)\mapsto x\triangleright y$, is an operation 
%	such that the maps $\varphi_x\colon
%	X\times X\to X$, given by $y\mapsto x\triangleright y$, are bijective for
%	each $x\in X$, and $x\triangleright
%	(y\triangleright z)=(x\triangleright y)\triangleright (x\triangleright z)$
%	for all $x,y,z\in X$.
%\end{definition}
%
%\begin{exercise}
%	Let $X$ be a set with an operation $X\times X\to X$, $(x,y)\mapsto x\triangleright y$. 
%	Prove that the map $r\colon X\times X\to X\times X$, $r(x,y)=(x\triangleright y,x)$,
%	is a solution of the set-theoretical braid equation
%	\[
%	(r\times\id)(\id\times r)(r\times\id)
%	=(\id\times r)(r\times\id)(\id\times r)
%	\]
%	if and only if $(X,\triangleright)$ is a rack.
%\end{exercise}
%
%Racks produce braided vector spaces.  Let $X$ be a rack, $V=\C X$
%be the vector space with basis $\{x:x\in X\}$ and $c(x\otimes
%y)=x\triangleright y\otimes x$.  Then $(V,c)$ is a braided vector space. 
%
%
%\begin{example}
%	Let $G$ be a group and $X$ be a union of conjugacy classes. Then $X$ with
%	$x\triangleright y=xyx^{-1}$ for all $x,y\in X$ is a quandle. 
%\end{example}
%
%\begin{example}
%	Let $X=\Z/n$ and $x\triangleright y=2x-y$. Then $(X,\triangleright)$ is a rack.
%\end{example}
%
%\begin{example}
%	Let $X$ be an abelian group, $g\in\Aut(X)$ and $x\triangleright y=(\id-g)(x)+g(y)$. 
%	Then $(X,\triangleright)$ is a rack.
%\end{example}
%
%\begin{exercise}
%	\label{xca:2cocycle}
%	Let $X$ be a rack and $q\colon X\times X\to\C^{\times}$ be a map. Let $V=\C
%	X$ and 
%	\[
%		c(x\otimes y)=q(x,y)x\triangleright y\otimes x.
%	\]
%	Prove that $(V,c)$ is a braided vector space if and only if 
%	\begin{equation}
%		\label{eq:2cocycle}
%		q(x,y\triangleright z)q(y,z)=q(x\triangleright y,x\triangleright z)q(x,z)
%	\end{equation}
%	for all $x,y,z\in X$. 
%\end{exercise}
%
%
%\begin{example}	
%	Let $-X=(123)^{\Alt_4}$ be the quandle associated with the conjugacy class
%	of $(123)$ in the alternating group $\Alt_4$. The map $q\colon X\times
%	X\to\C$ given by
%	\begin{equation}
%		\label{eq:2cocycle}
%		\begin{array}{c|cccc}
%			& (243) & (123) & (134) & (142)\\
%			\hline (243) & \omega & \omega & \omega & \omega\\
%			(123) & \omega & \omega & -\omega & -\omega\\
%			(134) & \omega & -\omega & \omega & -\omega\\
%			(142) & \omega & -\omega & -\omega & \omega
%		\end{array}
%	\end{equation}
%	where $\omega\in\C$ is a primitive $n$-th root of $1$, is a $2$-cocycle of
%	$X$. 
%\end{example}
%
%\begin{example}
%	\label{exa:FK}
%	Let $n\geq3$ and $X_{n}=(12)^{\Sym_{n}}$.  The map 
%	$\chi\colon X\times X\to\C$ given by
%	\[
%		\chi(\sigma,\tau)=\begin{cases}
%		1 & \text{if }\sigma(i)<\sigma(j),\\
%		-1 & \text{otherwise,}
%	\end{cases}
%	\]
%	where $\tau=(ij)$, $i<j$, is a $2$-cocycle of $X_{n}$.
%\end{example}
%
%The braided vector spaces of Exercise~\ref{xca:2cocycle} are called of type $(X,q)$. 
%Braided vector spaces coming from 
%Yetter-Drinfeld modules over groups are braided vector spaces of type $(X,q)$,
%for some rack $X$ and some $q$ such that~\eqref{eq:2cocycle} holds.



\section{Nichols algebras}

Braided vector spaces produce representations of the braid group. 
The braid group $\B_n$ has generators
$\sigma_{1},...,\sigma_{n-1}$ and relations
\begin{align*}
	&\sigma_{i}\sigma_{i+1}\sigma_{i}=\sigma_{i+1}\sigma_{i}\sigma_{i+1} && \text{for }1\leq i\leq n-2,\\
	&\sigma_{i}\sigma_{j}=\sigma_{j}\sigma_{i} && \text{for \ensuremath{|i-j|>1}.}
\end{align*}

Let $(V,c)$ be a braided vector space, and $n\in\N$. Define
\[
c_{i}=\id_{V^{\otimes(i-1)}}\otimes c\otimes\id_{V^{\otimes(n-i-1)}}\in\Aut(V^{\otimes n})
\]
for $1\leq i\leq n-1$, i.e.,
\[
c_{i}\cdot(v_{1}\otimes\cdots\otimes v_{n})=v_{1}\otimes\cdots\otimes v_{i-1}\otimes c(v_{i}\otimes v_{i+1})\otimes v_{i+2}\otimes\cdots\otimes v_{n}.
\]
The operators $c_{i}$, $1\leq i\leq n-1$, satisfy the defining-relations of the
braid group and hence 
\[
	\rho_{n}:\B_{n}\to\Aut(V^{\otimes n}),
	\quad
	\sigma_i\mapsto c_i,
\]
is a representation of $\B_{n}$ on $V^{\otimes n}$.


The symmetric group $\Sym_{n}$ can be presented with generators 
$\tau_{1},...,\tau_{n-1}$ and relations
\begin{align*}
	&\tau_{i}\tau_{i+1}\tau_{i}=\tau_{i+1}\tau_{i}\tau_{i+1} &  & 1\leq i\leq n-2,\\
	&\tau_{i}\tau_{j}=\tau_{j}\tau_{i} &  & \ensuremath{|i-j|>1},\\
	&\tau_{i}^{2}=1 &  & \leq i\leq n-1.
\end{align*}
Thus there exists a surjection $\B_{n}\to\Sym_{n}$ defined by
$\sigma_{i}\mapsto\tau_{i}$. 

\begin{lemma}[Matsumoto]
	There exists a set-theoretical section $\mu:\mathbb{S}_{n}\to\mathbb{B}_{n}$,
	$\tau_{i}\mapsto\sigma_{i}$, such that $\mu(xy)=\mu(x)\mu(y)$ if
	$\mathrm{length}(xy)=\mathrm{length}(x)+\mathrm{length}(y)$. 
\end{lemma}

\begin{proof}
	See 
\end{proof}

Let $(V,c)$ be a braided vector space. The \emph{Nichols algebra} of $(V,c)$ is 
\[
\NA(V,c)=\C\oplus V\oplus \bigoplus_{n\geq2}V^{\otimes n}/\ker\mathfrak{S}_n,
\]
where 
\[
\mathfrak{S}_{n}=\sum_{\sigma\in\Sym_{n}}\rho_{n}\mu(\sigma).
\]
is the \emph{quantum symmetrizer} of degree $n$.

\begin{example}
	Let us compute some quantum symmetrizers:
	\begin{align*}
		&\mathfrak{S}_2 = \id+c,\\
		&\mathfrak{S}_3 = \id+c_{1}+c_{2}+c_{1}c_{2}+c_{2}c_{1}+c_{1}c_{2}c_{1},
	\end{align*}
	where $c_1=c\otimes\id$ and $c_2=\id\otimes c$. 
\end{example}

Now two basic examples of Nichols algebras:

\begin{example}
	Let $V$ be a complex vector space and let $\tau:V\otimes V\to V\otimes V$
	be the linear map defined by $x\otimes y\mapsto y\otimes x$. The Nichols
	algebra of the braided vector space $(V,\tau)$ is the Symmetric Algebra
	$S(V)$. The Nichols algebra of the braided vector space $(V,-\tau)$ is the
	Exterior Algebra $\Lambda(V)$.
\end{example}

\section{Examples and computations}

We first present some very naive scripts to compute dimensions and generating
relations up to some given degree. To improve our function we use the package
ToolsForHomalg, written by Barakat, Gutsche and Lange-Hegermann.

\begin{lstlisting}
gap> LoadPackage( "ToolsForHomalg" );
\end{lstlisting}

Let us first start with the braiding. 

The following function computes the quantum symmetrizer recursively:
\begin{lstlisting}
recursive_symmetrizer := FunctionWithCache( function(c, n)
  local i, m, d;

  d := Sqrt(Size(c));

  if n=1 then
    return IdentityMat(d,d);
  fi;

  m := IdentityMat(d^n, d^n);
  for i in [1..n-1] do
    m := m+several_c(c, i, n);
  od;

  return m*KroneckerProduct(IdentityMat(d,d), recursive_symmetrizer(c,
n-1));
end : Cache := "crisp" );
\end{lstlisting}

\begin{example}[Yamane]
Let us start with a complex vector space $V$ with basis $\{a,b\}$. Let
$q\in\C$ be a primitive cubic root of $1$ and let $c\in\GL(V\otimes V)$ be
given by
\begin{align*}
	c(a\otimes a)=qb\otimes a,&&
	c(a\otimes b)=qa\otimes a,&&
	c(b\otimes a)=qb\otimes b,&&
	c(b\otimes b)=qa\otimes b.
\end{align*}
Then $\dim\NA(V,c)=108$. The Hilbert series is
\begin{multline*}
H(t)=1+2t+ 4t^2+ 6t^3+ 9t^4+ 12t^5+ 13t^6
+ 14t^7+ 13t^8+ 12t^9+ 9t^{10}\\+ 6t^{11}+ 4t^{12}+ 2t^{13}+t^{14}.
\end{multline*}
There are two relations of degree three and two of degree six:
\begin{align*}
&a^2b -q^2ab^2 -q^2ba^2 + b^2a=
a^3 + aba -qab^2 -qba^2 + bab + b^3=0,\\
&a^5b + a^4ba + a^3ba^2 + a^2ba^3 + aba^4 + ba^5=
a^6=0.
\end{align*}
This braided vector space is of diagonal type with respect to the basis
$\{a+b,a-b\}$. %The braiding matrix has entries q, q, -q, -q and Dynkin diagram
% q -q^2 -q. 
%The Nichols algebra is a special case of the one parameter
%family with Dynkin diagram q r^{-1} r, where $q$ is a primitive third root
%of 1. 
This family was first found by Yamane as his $\Z_3$-graded quantum group.
\end{example}

Now we show two examples not related to Yetter-Drinfeld modules over groups.

\begin{example}[Rowell]
Let $V$ be a vector space with basis $\{a,b\}$. Let
$\lambda=\frac{-1+i}{2}$ and let $c$ be the linear map given by
\begin{align*}
c(a\otimes a)=\lambda a\otimes a-\lambda b\otimes b, && 
c(a\otimes b)=\lambda a\otimes b-\lambda b\otimes a,\\
c(b\otimes a)=\lambda a\otimes b+\lambda b\otimes a, &&
c(b\otimes b)=\lambda a\otimes a+\lambda b\otimes b.
\end{align*}

There are two relations in degree two:
\[
	ab=iba,\quad
	ia^2=b^2.
\]
Then $\dim\NA(V,c)=5$. The Hilbert series is 
\[
    H(t)=1+2+2t^2.
\]
A linear basis for $\NA(V,c)$ is $\{1,a,b,a^2,ab\}$. 
\end{example}

\begin{example}[Rowell]
	
\end{example}

%
%\begin{block}
%	A braided vector space yields a special type of (braided) Hopf algebra called
%	the \textbf{Nichols algebra} of $(V,c)$. To define Nichols algebras we need the Artin braid 
%	group $\B_{n}$. This group can be presented with generators 
%	$\sigma_{1},...,\sigma_{n-1}$ and relations
%	\begin{align*}
%		&\sigma_{i}\sigma_{i+1}\sigma_{i}=\sigma_{i+1}\sigma_{i}\sigma_{i+1} && \text{for }1\leq i\leq n-2,\\
%		&\sigma_{i}\sigma_{j}=\sigma_{j}\sigma_{i} && \text{for \ensuremath{|i-j|>1}.}
%	\end{align*}
%	Recall that the symmetric group $\Sym_{n}$ can be presented with generators 
%	$\tau_{1},...,\tau_{n-1}$ and relations
%	\begin{align*}
%		&\tau_{i}\tau_{i+1}\tau_{i}=\tau_{i+1}\tau_{i}\tau_{i+1} &  & \text{for }1\leq i\leq n-2,\\
%		&\tau_{i}\tau_{j}=\tau_{j}\tau_{i} &  & \text{for \ensuremath{|i-j|>1},}\\
%		&\tau_{i}^{2}=1 &  & \text{for }1\leq i\leq n-1.
%	\end{align*}
%	Hence there exists a surjection $\B_{n}\to\Sym_{n}$ defined by
%	$\sigma_{i}\mapsto\tau_{i}$. 
%
%	\begin{lem*}[Matsumoto]
%		There exists a set-theoretical section $\mu:\mathbb{S}_{n}\to\mathbb{B}_{n}$,
%		$\tau_{i}\mapsto\sigma_{i}$, such that $\mu(xy)=\mu(x)\mu(y)$ if
%		$\mathrm{length}(xy)=\mathrm{length}(x)+\mathrm{length}(y)$. %The map $\mu$ is
%		%called the \textbf{Matsumoto section}.
%	\end{lem*}
%
%	\begin{rem*}
%		Let $(V,c)$ be a braided vector space, and $n\in\N$. Define
%		\[
%		c_{i}=\id_{V^{\otimes(i-1)}}\otimes c\otimes\id_{V^{\otimes(n-i-1)}}\in\Aut(V^{\otimes n})
%		\]
%		for $1\leq i\leq n-1$, i.e.,
%		\[
%		c_{i}\cdot(v_{1}\otimes\cdots\otimes v_{n})=v_{1}\otimes\cdots\otimes v_{i-1}\otimes c(v_{i}\otimes v_{i+1})\otimes v_{i+2}\otimes\cdots\otimes v_{n}.
%		\]
%		The operators $c_{i}$ ($1\leq i\leq n-1$) satisfy the defining-relations of the
%		braid group and hence $\rho_{n}:\B_{n}\to\Aut(V^{\otimes n})$,
%		defined by $\rho_{n}(\sigma_{i})=c_{i}$, is a representation of
%		$\B_{n}$ into $V^{\otimes n}$.
%	\end{rem*}
%\end{block}

\end{document}
